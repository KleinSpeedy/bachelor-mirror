%!TeX root = ./../Bachelorarbeit.tex

%##########################################################
% Inhalt
%##########################################################

\DeclareAcronym{htwk}{
    short = HTWK,
    long = Hochschule für Technik{,} Wirtschaft und Kultur Leipzig
}

\DeclareAcronym{fim}{
    short = FIM,
    long = Fakultät Informatik und Medien
}

\DeclareAcronym{can}{
    short = CAN,
    long = Controller Area Network
}

\DeclareAcronym{ram}{
    short = RAM,
    long = Random Access Memory
}

\DeclareAcronym{arm}{
    short = ARM,
    long = Advanced RISC Machines
}

\DeclareAcronym{cpu}{
    short = CPU,
    long = Central Processing Unit
}

\DeclareAcronym{jtag}{
    short = JTAG,
    long = Joint Test Action Group
}

\DeclareAcronym{soc}{
    short = SoC,
    long = System-on-a-Chip
}

\DeclareAcronym{ipc}{
    short = IPC,
    long = inter process communication
}

\DeclareAcronym{vm}{
    short = VM,
    long = Virtual Machine
}

\DeclareAcronym{vmm}{
    short = VMM,
    long = Virtual Machine Monitor
}

\DeclareAcronym{kvm}{
    short = KVM,
    long = Kernel-based Virtual Machine
}

\DeclareAcronym{uart}{
    short = UART,
    long = Universal Asynchronous Receiver/Transmitter
}

\DeclareAcronym{usart}{
    short = USART,
    long = Universal Synchronous/Asynchronous Receiver/Transmitter
}

\DeclareAcronym{gpio}{
    short = GPIO,
    long = General Purpose Input/Output
}

\DeclareAcronym{rtos}{
    short = RTOS,
    long = Real Time Operating System
}

\DeclareAcronym{api}{
    short = API,
    long = Application Programming Interface
}

\DeclareAcronym{cmsis}{
    short = CMSIS,
    long = Common Microcontroller Software Interface Standard
}

\DeclareAcronym{hal}{
    short = HAL,
    long = Hardware Abstraction Layer
}
