%!TeX root = ./../Bachelorarbeit.tex

%##########################################################
% Inhalt
%##########################################################

\clearpage
\chapter{Auswertung und Ausblick}

Dieser Abschnitt stellt den Schlusspunkt der Arbeit dar. In diesem Abschnitt (und im Diskussionteil) erwartet der Leser, 
dass er Antworten auf die in der Einleitung formulierten Fragestellungen findet und sich vergewissert, 
dass diese wirksam verteidigt wurden und mit der von Ihnen formulierten \enquote{These} übereinstimmen.

Die Hauptziele der Diskussion bestehen darin, eine Analyse Ihrer gesammelten Ergebnisse zu präsentieren, 
Ihre Ergebnisse angemessen darzustellen und eine Einschätzung der Bedeutsamkeit Ihrer Arbeit zu geben. Beachten Sie hier,
den Unterschied zur Diskussion im vorherigen Kapitel. Diskutieren Sie hier vor allem den Wert und die Bedeutung Ihrer Ergebnisse, auf Basis der 
Interpretationen aus dem vorherigen Kapitel. Beziehen Sie sich gern auch auf Ihr beschriebenes Problem und Ihr Ziel.

Jede wichtige Schlussfolgerung, die Sie im \enquote{Ergebnisteil} gezogen haben, 
muss hier erneut behandelt werden. Eine gewisse Anzahl von Wiederholungen ist unvermeidlich. Darüberhinaus, 
die Ergebnisse anderer Forschungsarbeiten, müssen mit eindeutigen Verweisen auf auffindbare Literaturquellen versehen sein.

Der Fazit-Teil kann als eine kurze Zusammenfassung Ihrer Diskussion betrachtet werden. 
Der Leser muss sich hier schnell einen Überblick über den Inhalt und die Bedeutung der Arbeit als Ganzes verschaffen.

Dieser Abschnitt soll einen Überblick präsentieren und dient dazu, dem Hauptteil Ihrer Arbeit den letzten Schliff zu geben. 
Die Schlussfolgerung kann auch Hinweise auf ein mögliches zukünftiges Werk enthalten.

\underline{Aber \textbf{mindestens} folgende Punkte}\\
Inhalt des fünften Kapitels (im Allgemeinen):
\begin{itemize}
    \item Reflektieren Sie hier nun die Ergebnisse der Tests aus dem letzten Kapitel
    \item Ordnen Sie diese in den Gesamtkontext ein... gut/schlecht? Was kann man verbessern/anders machen? Schätzen Sie auch ab was Veränderungen bringen könnten.
    \item Was wurde durch die Ergebnisse gezeigt? Geben Sie eine bewertende (selbstkritische) Aussage ab zu Ihrem Schaffen
    \item Geben Sie einen Ausblick was nun folgen sollte/könnte.
\end{itemize}
