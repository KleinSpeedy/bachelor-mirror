%!TeX root = ./../Bachelorarbeit.tex

%##########################################################
% Inhalt
%##########################################################

\clearpage
\chapter{Auswertung und Ausblick}

Die Emulation eines 32-Bit Mikrocontrollers durch das Programm QEMU bietet
viele Möglichkeiten, den aufwändigen Prozess der Embedded Software Entwicklung
zu vereinfachen.
Die Ausführung einer Embedded Software Anwendung im Kontext einer emulierten
Architekturplattform kommt einer echten Hardwareumgebung bereits sehr nahe.
Allerdings ist der Implementationsaufwand hoch.
Bereits wenig komplexe Mikrocontroller-Peripherien benötigen ein umfangreiches
Verständnis des gesamten Systems.

\section{Auswertung}

\subsection{Emulationsergebnisse}

Die Erweiterung von QEMU durch den STM32F429-\ac{soc} und die \ac{gpio}
Peripherie ermöglicht bereits jetzt eine zuverlässige Emulation des Systems.
Im Vergleich mit der QEMU-\ac{edi} Erweiterung war die Device-Erweiterung
komplexer, aber auch weniger fehleranfällig.
Sie ermöglicht die fast volle Emulation STM32F429-\ac{soc}.
Je vollständger die Implementation der Mikrocontroller-Peripherie, desto besser
kann der Test auf realer Hardware ersetzt werden.
Wie in Kapitel \ref{sec:impl-tests-emulation} gesehen werden konnte, ist eine
Ausführung einfacher Testanwendungen fast ohne Einschränkungen möglich.
Hier ist das Alter des QEMU-Projekts ein weiterer Vorteil.
Es existieren bereits viele Implementationen, die direkt für einen
Mikrocontroller genutzt werden können.
Sollte eine direkte Nutzung einer Peripherie nicht möglich sein, kann sie meist
dennoch als Vorlage für eine neue Implementation verwendet werden.
Häufig ähneln sich die Konzepte der Schnittstellen.

Das QEMU-\ac{edi} Projekt ist ebenso in der Lage, einfache Peripherie schnell
zu emulieren.
Der Ansatz der \ac{ipc}-Kommunikation ist gut dafür geeignet, kurzfristige,
schnelle Testimplementationen von Mikrocontrollern und der dazugehörigen
Peripherien zu entwickeln.
Eine unvollständige Peripherie-Abdeckung ist bei diesem Ansatz kein großes
Problem, da es einfach ist, ein neues Skript für eine neue Peripherie zu
schreiben.
Die Austauschbarkeit emulierter und hardwarebasierter Testanwendungen erhöht die
Entwicklungsgeschwindigkeit.
Die Nutzung der gleichen Schnittstellen kann beim Design von
Softwarearchitektur helfen, da die Testanwendung sowohl in der Emulation als
auch auf der echten Hardware die gleiche \enquote{logische Struktur} haben.
Es besteht ein hoher Freiheitsgrad für Implementationen.
Er wird lediglich beschränkt durch das Konzept der \ac{ipc}-Kommunikation.
Diese ist aber bereits in viele moderne Programmiersprachen gut integriert.

\subsection{Problemdiskusion}

Obwohl die Ergebnisse der QEMU-Device Erweiterung bereits eine gute Abbildung
der echten Hardware darstellen, so ist der Implementationsaufwand dennoch hoch.
Die QEMU-\acp{api} und Kommandozeilenargumente stellen mächtige Werkzeuge dar.
Allerdings können sie auch den Aufwand einer Implementation ungewollt erhöhen.
Als Beispiel hierfür sind die \textit{Properties}.
Es war nicht einfach möglich die Werte eines Objekts dynamisch zu verändern,
wie in Kapitel \ref{sec:impl-qemu-reset} dargestellt wurde.
Das schränkt den Implementationsfreiraum ein und führt zur Erhöhung der
Softwarekomplexität.
Ein weiteres Problem ist die Unvollständigkeit der Peripherie-Implementationen.
Wie in Kapitel \ref{sec:impl-tests-emulation} zu sehen, hat das Fehlen einer
wichtigen Mikrocontroller-Komponente berits große Auswirkungen auf die
Vollständigkeit der Emulation.
Je \enquote{zentraler} die Funktion der Peripherie, desto eher sollte sie
implementiert werden.
Das Alter und die Komplexität der QEMU Softwarearchitektur wirken sich
ebenfalls restriktiv auf zukünftige Implementationen aus.
Ein Beispiel hierfür ist die Nutzung der C Programmiersprache.
Es werden moderne Programmierparadigmen, wie Objektorientierte Programmierung
oder die Definition und Implementation von Schnittstellen in QEMU angewandt.
Allerdings sind diese bei weitem nicht so stark ausgeprägt wie in einer
moderneren Programmiersprache wie C++ oder Rust.
Die Fehleranfälligkeit von Programmiersprachen ohne starke Typisierung, wie
beispielsweise C, ist vergleichsweise höher als bei Sprachen mit starker
Typisierung\cite{GithubProgrammingLanguagesStudy}.
Eine weitere Hürde bei der QEMU-Device Erweiterung ist die Bereitstellung von
\enquote{Simluationsdaten}.
Es ist in QEMU nicht ohne weiteres möglich, Testdaten für eine Peripherie
bereitzustellen.
Für einfache Schnittstellen wie \ac{uart} ist dies noch kein großes Problem.
Je komplexer die Peripherie wird, desto höher sind die Anforderungen an eine
vollständige Emulation.
Es bedingt beispielsweise die Implementation komplexer
\textit{Device-Backends}, welche den Gesamtaufwand der Emulation deutlich
erhöhen.

Auch die Nutzung moderner Programmierparadigmen ist keine Garantie für eine
Verringerung der Komplexität.
Ein Beispiel hierfür ist die im \ac{edi} Demo-Projekt eingeführte
Abstraktionsebene für die Austauschbarkeit der darunter liegenden
Implementation.
Diese wird ermöglicht durch die Definition von Schnittstellen und damit der Austauschbarkeit einer Implementation.
% und damit austauschbarkeit der Implemntation
Es erhöht aber auch die Komplexität der Gesamtanwendung.
Sie verfehlt darüber hinaus das verfehlt aber das Ziel der
vollständigen Abstraktion der echten Hardware durch eine emulierte Umgebung.
Das Ziel einer Emulation sollte es sein, Software ohne jegliche Anpassung
ausführen zu können, egal ob in einer emulierten Umgebung oder auf echter
Hardware.
Die Abstraktion der darunter liegenden Implementation ermöglicht zwar schnelles
\enquote{Prototyping}, allerdings wird die Software dadurch schwerer
verständlich.
Die Komplexität von \ac{ipc}-Kommunikation kann ebenfalls eine Einstiegshürde
sein.
Sowhol die Warbarkeit, als auch Erweiterbarkeit von Software wird dadurch im
schlimmsten Fall ebenfalls vermindert.
Wichtig ist hierbei eine klare Definition der verwendeten Schnittstellen und die
Nutzung ausgereifter, möglichst fehlerfreier Software.
Im Falle der QEMU-\ac{edi} Erweiterung wird dies komplizierter, da das Projekt
auf den Stand der Version 4.2.0 festgesetzt ist.
Das Projekt muss ohne Bugfixes und neue Features auskommen.
Ein Update auf eine neuere Version wäre sehr aufwändig.

\section{Ausblick}

Die Implementation einer \ac{gpio} Peripherie ist nur ein erster Schritt für
die vollständige Emulation eines Mikrocontroller.
\newline
Im nächsten Schritt sollte zuerst zentrale, wichtige Peripherie implementiert
werden.
Ein Beispiel hierfür ist \ac{rcc} oder des Interrupt Watchdog, beziehungsweise
Window Watchdog.
Darüber hinaus sollten die bestehenden Peripherien vervollständigt werden.
Ein Beispiel hierfür die \enquote{alternative Funktion} der
\ac{gpio}-Peripherie.
Sind diese Peripherie Geräte implementiert, sollten die  Ethernet und \ac{can}
Schnittstellen folgen.
Beide sind von zentraler Beudetung im Kontext moderener Embedded Software.
\newline
Abseits davon wäre die Bereitstellung von Testdaten eine wichtige Erweiterung.
Eine grafische Darstellung eines Entwicklungsboards in Verbindung mit der
Emulation durch QEMU könnte helfen, das Zusammenspiel zwischen Mikrocontroller
udes nd Peripherie besser zu veranschaulichen.
Darüber hinaus ist die einfache Integration in automatische Testgenerierung,
beziehungsweise Integrationtesting ein breites Feld, was von den breiten
Möglichkeiten der Emulation stark profitieren würde.
