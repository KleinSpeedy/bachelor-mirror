%!TeX root = ./../Bachelorarbeit.tex

%##########################################################
% Inhalt
%##########################################################

\clearpage
\chapter{Konzeption}

Jedes Experiment/oder Softwarekonzeption sollte so detailliert beschrieben werden, dass ein anderer Wissenschaftler die Arbeit durchführen und genau das gleiche Ergebnis erzielen könnte, wie Sie. 
Mit anderen Worten: Ihre Forschung sollte reproduzierbar sein.

Jeder Faktor oder jede Überlegung, die für den Erfolg eines Ihrer Experimente eine wichtige Rolle gespielt hat, muss erwähnt werden.

All diese Punkte schließen natürlich auch ein, dass irgendwie getroffene Abwägungen oder Entscheidungen aufgezeigt und begründet sein müssen. Warum diese Art der Programmierung? Warum dieses Framework?
Alternativen? Am Ende muss für eine lesende Person klar herauskommen, warum Sie Ihre Lösung so konzipiert haben, wie Sie es getan haben und es zudem ein schlüssiger Entscheidungsweg ist. Keine Willkür.

\underline{Aber \textbf{mindestens} folgende Punkte}\\
Inhalt des dritten Kapitels (im Allgemeinen):
\begin{itemize}
    \item Basierend auf dem Fall, dass Sie eine (prototypische) Implementation vornehmen, sollten Sie hier nun diese konzipieren
    \item Wie haben Sie vor die Implementation vorzunehmen? Besonderheiten? Überlegungen? Einflüsse? Probleme?
    \item Gießen Sie Ihre Idee zunächst in ein theoretisches Konstrukt $\rightarrow$ Kapitel Vier ist der Beweis, dass Ihre Überlegungen (nicht) funktionieren.
\end{itemize}
