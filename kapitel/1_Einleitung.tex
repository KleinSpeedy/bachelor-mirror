%!TeX root = ./../Bachelorarbeit.tex

%##########################################################
% Inhalt
%##########################################################
\pagenumbering{arabic}
\chapter{Einleitung}

Eingebettete Systeme sind heutzutage allgegenwärtig.
Von einem intelligenten Heizungsthermostat integriert in ein System zur Haussteuerung,
über den Staubsaugroboter der mittels WiFi-Kommunikation durch die App auf dem
Smartphone gesteuert werden kann, hin zu Satelliten und selbstfahrenden Autos.
All dies benötigt Eingebettete Systeme und deren Software Applikationen.

\section{Problemstellung}

Bei der Entwicklung von Software für eingebettete Systeme gibt es mehrere Einflussfaktoren,
die den Prozess deutlich erschweren.
Im Vergleich zur \textit{Web-} oder \textit{Desktop} Anwendungs-Programmierung
sind die genutzten Platformen häufig sehr spezifisch.
Elektronische Baugruppen werden oft individuell zusammengestellt und auf ihre
unterschiedlichen Anwendungsbereiche angepasst.
Diese Vielfalt wird begleitet durch eine große Anzahl unterschiedlicher Mikroprozessor
Platformen und Architekturen.
Darüber hinaus gibt es selbst für einzelne Architekturen wie \textit{ARM}
oder \textit{RISC-V} von Hersteller zu Hersteller
verschiedene Mikrocontroller-Variationen.
Diese Variationen unterscheiden sich zum Bespiel bei der Menge an Ressourcen, wie
Flash- und \ac{ram} Speicher und der Ausstattung mit verschiedenen Peripherie Baugruppen,
wie Ethernet oder \ac{can}.
Software für Eingebettete Systeme muss also individuell auf unterschiedliche Platformen
angepasst werden und ist meist ebenso unterschiedlich wie die Plattformen auf denen
sie betrieben wird.

Darüber hinaus eignen sich Mikrocontroller nicht zur Entwicklung von Software.
Ressourcen und Rechenleistung sind deutlich geringer als beispielsweise auf einem
Heim-Computer.
Viele Anwendungen sind darauf ausgelegt möglichst energiesparend zu sein, da meist
keine dauerhafte Stromquelle zur Verfügung steht.
CPU Architektur der Ziel- und Entwicklungsplattform sind daher oft unterschiedlich,
was zur Nutzung plattformunabhängiger \textit{Toolchains} zur \textit{Cross-Kompilierung}
führt.
Software für eingebettete Systeme kann also nicht einfach auf der
Entwicklungsplattform ausgeführt werden, da die Plattformen meist auf
Instruktionsebene inkompatibel sind.

Fehler die nur bei der spezifischen Architektur auftreten können, sind durch
diese Inkompatibilität schwerer zu identifizieren.
Zur Vorbeugung dieser Probleme gibt es zwar Werkzeuge wie statische Code-Analyse,
diese können aber nicht alle Fehler finden oder beheben. 
Um die vollständige Korrektheit der Implementation zu testen, muss die Anwendung
also auf der Zielplattform ausgeführt werden. \newline
Zur Fehlersuche während der Ausführung des Programms wird meist extra Hardware benötigt,
welche auf integrierte Schnittstellen, wie zum Beispiel \ac{jtag}, zugreift,
um das Programm anzuhalten und den Zustand an spezifischen Zeitpunkten zu inspizieren.
Sind diese Schnittstellen oder die extra Hardware nicht vorhanden wird die Fehlersuche
deutlich aufwendiger und schwieriger, bei manchen Anwendungen sogar unmöglich.
Hinzukommt die Zugänglichkeit des Systems.
Die elektronischen Baugruppen und Mikrocontroller könnten sich in einem
versiegelten Gehäuse befinden oder sind möglicherweise gar nicht zugänglich,
beispielsweise bei Satelliten.

\section{Zielstellung}

Eine Möglichkeit den vielfältigen Problemen bei der Entwicklung von Software
für eingebettete Systeme zu begegnen, ist die Emulation der Zielplattform in Verbindung
mit der Simulation der verwendeten Peripherie.
Man umgeht damit das Problem der Inkompatibilität von Ziel- und Entwicklungsplattform.\newline
(Verweis Web-Engineering Vorlesung - Docker etc -> ähnliches Prinzip).

Ziel dieser Abschlussarbeit ist es, eine 32-Bit Mikrocontroller Plattform mitsamt
ihrer Peripherie zu emulieren, um die dafür entwickelte Software Applikation
virtualisiert auf der Entwicklungsplattform ausführen zu können.

Als ersten Schritt soll dafür ein geegneites Werkzeug zur Emulation der \ac{cpu}
ausgesucht werden.
Dieses Werkzeug sollte die folgenden Eigenschaften
\begin{itemize}
    \item Vollständigkeit der Zielplattform,
    \item Einfachheit der Nutzung im Kontext der Entwicklung,
    \item Performanz der ausgeführten Applikation und Anwendung und
    \item Erweiterbarkeit der Implementation
\end{itemize}
möglichst vollständig aufweisen.

Aufgrund der, in der Problemstellung genannten, Vielfältigkeit von Mikrocontrollern
soll im Falle einer unvollständigen Emulation eine Möglichkeit entwickelt werden,
wie möglichst viele Peripherie-Baugruppen einfach mittels Software simuliert werden
können.
Nach Vorhergehender Bewertung der Peripherie nach
\begin{itemize}
    \item Wichtigkeit im Kontext der Anwendung,
    \item Komplexität der Integrierung,
    \item Heterogenität,
    \item Vorhandensein
\end{itemize}
soll eine Baugruppe exemplarisch implementiert werden.

Anschließend soll die Vollständigkeit der Lösung mithilfe mehrerer exemplarischer
Beispiel Anwendungen getestet und bewertet werden.
(-> welche Anwendungen ? -> welche Kriterien ?)

\newpage

\section{Methodik}

Im Methodik-Abschnitt (eine eigene Section) einer Arbeit stellen Sie die Methoden dar,
die Sie verwendet haben oder vorhaben zu verwenden, um die Fragestellung zu beantworten. 
In diesem Abschnitt erläutern Sie, welche Methoden umgesetzt wurden (oder auch angedacht),
um die Hypothesen zu testen und die Fallstudie durchzuführen. 
Welche Methoden für Ihre Arbeit geeignet sind, ist auch ein wesentlicher Punkt Ihres
angestrebten Titels und somit ein zu bewertender Teil dieser Arbeit. 

\underline{Aber \textbf{mindestens} folgende Punkte}\\
Inhalt des ersten Kapitels (im Allgemeinen):
\begin{itemize}
    \item Thematische Einführung, umreißen des Themas \dots wo befinden wir uns?
    \item Herausstellen des Problems und warum es eines ist was gelöst werden sollte.
    \item Herausstellen der tatsächlichen Ziel-/Fragestellung, welche bearbeitet werden wird.
    \item Eine Abgrenzung was diese Arbeit ist und was sie nicht ist.
    \item Stand der Forschung \dots Was existiert bereits? Wie gut passt das auf das Problem?
    \begin{itemize}
        \item Referenzen mittels cite: \cite[S.~111]{jsch2011} \cite[S.~27f]{Tane2014}
    \end{itemize}
    \item Eine abgeleitete Methodik $\rightarrow$ basierend auf der Literatur und Ihrem Wissen: wie haben Sie nun vor Ihr Problem zu lösen? Vllt durch \cite{9429985}?
\end{itemize}
