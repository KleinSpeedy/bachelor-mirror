%!TeX root = ./../Bachelorarbeit.tex

%##########################################################
% Inhalt
%##########################################################
\pagenumbering{arabic}
\chapter{Einleitung}

Eingebettete Systeme kann man heutzutage in nahezu jedem elektronischen Gerät
finden.
Von einem intelligenten Heizungsthermostat integriert in ein System zur
Haussteuerung, über den Staubsaugroboter der mittels WiFi-Kommunikation durch
die App auf dem Smartphone gesteuert werden kann, hin zu Satelliten und
selbstfahrenden Autos.
All dies benötigt eingebettete Systeme und deren Software Applikationen.
Die Entwicklung dieser wird allerdings von verschiedenen Herausforderungen
begleitet, welche im Gegensatz zur Entwicklung von \textit{Web-} oder
\textit{Desktop} Anwendungen nicht auftreten.\newline
Die große Vielfalt an Anwendungsbereichen führt zu vielen verschiedenen,
individuell zusammengestellten elektronischen Baugruppen, welche auf einer
großen Auswahl verschiedener Mikroprozessor Plattformen und Architekturen
basieren.\newline
Darüber hinaus gibt es selbst für einzelne Architekturen wie beispielsweise
\textit{ARM} oder \textit{RISC-V} von Hersteller zu Hersteller verschiedene
Mikrocontroller Varianten.
Diese unterscheiden sich zum Beispiel in der Menge an Ressourcen, wie Flash-
und \acs{ram} Speicher und der Ausstattung mit verschiedenen
Peripherie Baugruppen, wie Ethernet oder \acs{can}.\newline
Software für eingebettete Systeme muss also individuell auf unterschiedliche
Zielplattformen angepasst werden.

\section{Problemstellung}

Mikrocontroller eignen sich nicht zur Entwicklung von Software.
Sie besitzen deutlich weniger Speicher und Rechenleistung als beispielsweise
Workstations oder Laptops.
Auch die Anbindung nötiger Peripheriegeräte zur Bedienung wie Maus, Tastatur
oder Bildschirm ist nicht ohne weiteres möglich.
Darüber hinaus ist die CPU Architektur der Ziel- und Entwicklungsplattform in
der Regel unterschiedlich, was eine Inkompatibilität von Ziel- und
Entwicklungsplattform auf Maschinenbefehl-Ebene zur Folge hat.\newline
Aufgrund dieser Punkte können Anwendungen für eingebettete Systeme nicht
einfach auf der Entwicklungsplattform gebaut und ausgeführt werden.
Um die Firmware zu erstellen ist die Nutzung plattformunabhängiger
\textit{Toolchains} zur \textit{Cross-Kompilierung} nötig.
Dies verlangsamt den Entwicklungsprozess, da Firmware, zum Beispiel für Tests,
immer erst auf den Mikrocontroller \textit{geflasht} werden muss.
Das Flashen geschieht im Entwicklungsprozess meist unter Zuhilfenahme
von extra Hardware, welche genutzt wird, um die Firmware auf den Mikrocontroller
zu schreiben.\newline
Auch der direkte Zugriff auf das System ist nicht immer gewährleistet.
Zahlen des Statistischen Bundesamtes zeigen, dass im IT-Dienstleistungsbereich
im Jahr 2021 gut drei Viertel der Beschäftigten im Home-Office
arbeiteten\cite{DestatisHomeOffice}.
Der Mangel an Mikrocontrollern und anderen elektronischen Bauteilen kann
ebenfalls problematisch sein, weil nicht für jeden Entwickler ein Testgerät
bereitgestellt werden kann.\newline
Ein weiteres Problem besteht darin, dass Compiler spezifische Fehler aufweisen
können, welche bei Compilern für die Entwicklungsplattform möglicherweise nicht
auftreten\cite{DebGccBug}\cite{LaunchpadGccBug} oder in neueren Versionen schon
behoben wurden.
Diese Fehler zu identifizieren ist schwierig und sie nachzustellen zudem
sehr zeitaufwändig.\newline
Zur Fehlersuche während der Ausführung des Programms wird meist extra Hardware
benötigt, welche auf integrierte Schnittstellen, wie zum Beispiel
\acs{jtag}, zugreift.
Das laufende Programm kann anschließend pausiert werden, um den Zustand an
spezifischen Zeitpunkten zu inspizieren.
Sind diese Schnittstellen oder die extra Hardware nicht vorhanden wird die
Fehlersuche deutlich aufwändiger und schwieriger, bei manchen Anwendungen sogar
unmöglich.
Zur Vorbeugung dieser Probleme gibt es zwar Werkzeuge wie statische
Code-Analyse.
Diese können aber nicht alle Fehler finden oder beheben.
Um die Korrektheit der Implementation zu überprüfen, \textit{muss} die
Anwendung also zum Test unter Bedingungen, welche denen auf der Zielplattform
sehr nahe kommen ausgeführt werden.\newline
Eine Möglichkeit reale Bedingungen bei der Ausführung abzubilden, ist die
Emulation des Mikrocontrollers und der integrierten Peripheriebaugruppen.
Dies ermöglicht die lokale Ausführung und Inspektion der Anwendung auf der
Entwicklungsplattform während der Laufzeit.
Die Notwendigkeit für extra Hardware zum flashen und \textit{debuggen} der
Software wird reduziert.
Das erhöht die Entwicklungsgeschwindigkeit und macht Entwickler Hardware- und
ortsunabhängiger.

\section{Zielstellung}

Im Rahmen dieser Arbeit soll eine Herangehensweise vorgestellt werden, mithilfe
derer man Software für einen 32-Bit Mikrocontroller unabhängig von der
tatsächlichen Hardware entwickeln kann.
Zur Emulation von \acs{cpu} und \acs{soc} soll
das Programm \textit{QEMU} genutzt werden.\newline
Für diese Arbeit wurde das Mikrocontroller Modell \textbf{STM32F429} der
\textbf{STM32F4} Familie als Zielplattform ausgewählt.
Zum Zeitpunkt dieser Arbeit ist die Emulation der CPU möglich, es besteht aber
keine Implementation für den SoC in QEMU.
Es existieren allerdings Integrationen für Modelle verwandter Produktfamilien,
so zum Beispiel für die Mikrocontroller \textbf{STM32F405} oder
\textbf{STM32F205}\cite{QemuSTMDoku}.
Die Modelle der Produktfamilien \textbf{F4} und \textbf{F2} unterscheiden sich
nur geringfügig in Konfiguration und Speichergrößen und können daher als Basis
zur Emulation des gewählten Modells dienen.
Die Anzahl unterstützter Peripherie ist auch für die bereits integrierten
Modelle gering.
Um als reelle Testplattform zu dienen muss die Zielplattform möglichst
vollständig emuliert werden.\newline
Es soll daher für fehlende Peripherie eine prototypische Implementierung
entwickelt werden.
Die Implementierung aller fehlenden Peripheriebaugruppen ist sehr zeitaufwändig
und im Rahmen dieser Arbeit nicht möglich.
Im ersten Schritt soll dafür geeignete Peripherie ausgewählt werden.
Falls möglich sollen heterogene Peripheriebaugruppen identifiziert und
zusammengefasst werden.
Anschließend werden für die Integration der Peripherie zwei verschiedene
Ansätze betrachtet.\newline
Der erste Ansatz befasst sich mit der Erweiterung des bestehenden OpenSource-
Projekts durch die direkte Implementierung des Mikrocontrollers in QEMU.
Als Basis dafür dienen die bereits bestehenden Integrationen verwandter
Mikrocontroller Modelle.
Die bereits integrierte Peripherie soll, falls möglich, wieder verwendet und
anschließend um die neue, prototypische Implementierung erweitert
werden.\newline
Der zweite Ansatz soll die Möglichkeit untersuchen, fehlende Peripherie
mithilfe externer Prozesse zu emulieren.
Diese Prozesse kommunizieren mittels \acs{ipc} mit QEMU und sind
nicht direkter Bestandteil des Projekts.
Dieser Ansatz wurde 2021 auf einer Entwickler Konferenz vorgestellt
\cite{KplCodeDive}.
Als Basis dienen auch hier die bereits bestehenden Integrationen verwandter
Mikrocontroller Modelle.
Beide Ansätze sollen abschließend nach Komplexität und Erweiterbarkeit der
Implementierung, sowie mittels exemplarischer Beispiel-Anwendungen getestet und
bewertet werden.

\newpage
