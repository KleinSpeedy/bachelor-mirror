%!TeX root = ./../Bachelorarbeit.tex

%##########################################################
% Inhalt
%##########################################################

\clearpage
\chapter*{Kurzfassung}

Steigende Anforderungen an die Entwicklung von Embedded Software machen
effiziente Entwicklungsprozesse und gute Werkzeuge immer wichtiger.
Die Emulation von Mikrocontrollern kann helfen den manuellen Testaufwand zu
reduzieren und mehr Zeit für Entwicklung zu schaffen.

QEMU ist eine Anwendung zur Emulation verschiedener Hardware Architekturen.
In dieser Arbeit werden zwei verschiedene Ansätze für die Emulation des
STM32F429 Mikrocontrollers untersucht.

Der erste Ansatz beschäftigt sich mit der Erweiterung des bestehenden QEMU
Projekts durch den neuen STM32F429-SoC.
Das Gerät soll bestehende, kompatible Peripherien einbinden und durch eine neue
Implementation des GPIO Controllers erweitert werden.
Anschließend sollen Embeded Software Applikation die Korrektheit der
Implementation testen.
Der zweite Ansatz beschäftigt sich mit der Erweiterung QEMUs durch 
Möglichkeiten der Interprozesskommunikation.
Das \enquote{External Device Interface} wird prototypisch implementiert, auf
den STM32F429 angepasst und in QEMU getestet.

Die Emulation mittels QEMU erweist sich als vielversprechend.
Dennoch sind Implementationsaufwand und Komplexität hoch.
Konfigurationsfehler und eine unvollständige Peripherie Abdeckung kosten Zeit
bei der Fehlersuche, unabhängig vom gewählten Ansatz.
Der Fokus muss vorerst auf einer Vervollständigung der Emulation liegen.
Alternative Möglichkeiten zur Nutzung, wie automatische Tests, sollten
ebenfalls betrachtet werden.

\keywords{QEMU, Embedded Software, Emulation, Mikrocontroller, Programmierung}
