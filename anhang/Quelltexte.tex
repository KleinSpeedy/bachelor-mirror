%!TeX root = ./../Bachelorarbeit.tex

%##########################################################
% Inhalt
%##########################################################
\clearpage
\chapter{Anhang -\ Quelltexte}

\begin{minipage}{\linewidth}
\lstinputlisting[language=c,numbers=none,
                label={lst:qemu-gpio-memops},
                linerange={384-399,419-427},
                caption=Konfiguration der \ac{mmio}-Speicherregion des GPIO-Devices]
                {anlagen/qemu-device/stm32f429_gpio.c}
\end{minipage}

\begin{minipage}{\linewidth}
\lstinputlisting[language=c,numbers=none,
                label={lst:qemu-gpio-vmstates},
                linerange={364-382},
                caption=Definition der \texttt{VMStates} zum Laden/Speichern der Zustände eines Objekts]
                {anlagen/qemu-device/stm32f429_gpio.c}
\end{minipage}

\lstinputlisting[caption=STM32F429 SoC Header Datei,
                language=c,
                label={lst:qemu-soc-header}]
                {anlagen/qemu-device/stm32f429_soc.h}

\lstinputlisting[caption=STM32F429 GPIO Device Header Datei,
                language=c,
                label={lst:qemu-gpio-header}]
                {anlagen/qemu-device/stm32f429_gpio.h}

\lstinputlisting[caption=Skript zur Ausführung von QEMU mit Firmware,
                language=sh,
                label={lst:qemu-test-script}]
                {anlagen/qemu-device/stm32f429_gpio.h}
